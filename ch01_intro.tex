\chapter{Introduction}
\label{ch:intro}

In modern Natural Language Processing (NLP) research, there is great deal of
focus on sophisticated semantic tasks which require complex inference and
synthesis of knowledge. These include tasks like Question Answering (QA), where
computers must read and answer questions about passages
\cite{hirschman:2001:nle,allam:2012:ijrris,gupta:2012:ijca}, and Recognizing Textual Entailment
(RTE), where computers must decide whether a hypothesis utterance logically
follows (or can be inferred) from a given piece of text
\cite{dagan:2006:mlc,marelli:2014:semeval,bowman:2015:emnlp}. In the future,
these technologies could influence a wide range of industries: from threat
identification in defense, to fact checking in journalism, to
synthesis of knowledge in science and medicine.

Substantial progress has been made in systems which perform
sentential inferences in QA and RTE, especially as common benchmarks
and datasets have become available
\cite{dagan:2006:mlc,giampiccolo:2007:pascal,bentivogli:2009:tac,marelli:2014:semeval,bowman:2015:emnlp}.
Yet in most sophisticated, compositional model of semantics, systems must
ultimately consider the semantics of individual lexical items to form a
conclusion. This often requires an understanding about the different
relationships that can occur between lexical items. Consider the following
example:
\begin{quote}
  \label{ex:rte}
  Text (Antecedent): The bright girl reads a book.\\
  Hypothesis (Consequent): A smart child looks at pages of text.
\end{quote}
Any language processing system wishing to infer the second sentence from the
first must know quite a bit of information about these words: it must know that
girl is a kind of child (hypernymy), and that bright and smart have the same
meaning in this context (synonymy); that books contain pages of text
(meronymy), and that reading involves looking at pages of text (world knowledge).

Although significant progress has been made on the task of Recognizing Textual
Entailment, many of these systems ultimately depend on some lexical resources
\cite{beltagy:2014:semeval,bjerva:2014:semeval,lai:2014:semeval,marelli:2014:semeval,beltagy:2016:cl}.
Possibly the most famous lexical resource is WordNet \cite{miller:1995:acm},
which organizes the lexicon into a large ontology,
though many other resources also exist and are used
\cite{baker:1998:acl,baroni:2011:gems,baroni:2012:eacl,ganitkevitch:2013:naacl,jurgens:2012:semeval,levy:2014:conll,turney:2015:nle}.
Unfortunately, resources as expansive as WordNet are extremely expensive
to create, and as language is ever-changing, they are inevitably
always incomplete. As such, any dependence on manually constructed resources
represents one weak point in some Natural Language Understanding systems. Even
recent neural network approaches, which attempt to learn entailments without
explicitly depending on these resources, often cannot make entailment
predictions about words which were not in the training data
\cite{bowman:2015:emnlp,cheng:2016:emnlp,pavlick:2016:acl}.

Distributional Semantics offers one potential solution to these issues of lexical
coverage. Distributional Semantics takes inspiration from the famous quote:
``You shall know a word by the company it keeps'' \cite{firth:1957:la}. In
Distributional Semantics, representations of word meaning are automatically
induced by counting or modeling the {\em contexts} in which a word appears.
Distributional Semantics is often called Vector Space Models (VSMs) of
language, because words are represented as vectors a high-dimensional vector space.
Words with similar semantics will have similar vectors in this
space. Since VSMs do not require annotated corpora, they are used and
studied as an alternative or predictor of particular lexical resources
\cite{baroni:2012:eacl,erk:2010:gems,turney:2010:jair}.

In this thesis, we consider how VSMs can be leveraged to predict some of the
difficult lexical inferences necessary in RTE.  Namely, we present techniques
and models for predicting specific lexical relationships, entailments, and
substitutions using Distributional Semantics. In Lexical Relationship
detection, we must predict whether two words exhibit specific, fine-grained
linguistic relationships, like hypernymy (is-a relationships; e.g. \lit{cat}
and \lit{animal}) or meronymy (part-whole relationships; e.g. \lit{cat} and
\lit{tail}). In Lexical Entailment detection, we must predict a coarser
entailment label, such as simply entailing or non-entailing, without
fine-grained labels of linguistic relationship. In Lexical Substitution, we
must propose a context-specific synonym for a word in a given sentential
context, such that must consider a word's multiple meanings. 

We consider each of these tasks, in turn, and present novel models based on
distributional approaches of word meaning. Furthermore, we further demonstrate
the value of our efforts by integrating our techniques into a larger framework
of sentential semantics, and show improvements on a real, end-to-end RTE task,
especially compared to a model which uses only a fixed lexical resource.

While each of these tasks is approached and tested for empirical contributions,
we also aim to understand the improvements and benefits offered by each of our
models across the various tasks. To this end, each empirical contribution is
additionally complemented by an analysis of {\em why} the contribution is
attained. In this way, we hope we can show not just how distributional
semantics can enable lexical entailment, but how lexical entailment provides
insight about distributional representations, and the information contained
within them.



\section{Thesis Outline}

The remainder of this thesis is structured as follows:

In Chapter~\ref{ch:background}, we introduce the necessary background
information to understand this thesis, including a brief discussion
of Distributional Vector Space models and their construction. We also
briefly introduce each of the relevant tasks considered in this thesis,
including hypernymy detection, lexical relationship prediction, lexical
substitution, and Recognizing Textual Entailment.

In Chapter~\ref{ch:lexmem}, we consider the importance of experimental setup in
hypernymy detection, and propose an experimental setup which better measures
how models generalize to novel lexical items. We propose Asym, a Supervised
Distributional model for hypernymy detection, and show it outperforms prior
work in our experimental setup. Finally, we examine Asym analytically and show
it does not exhibit prototypicality behavior, where a model makes predictions
like ``cactus is an animal,'' only because animal is a prototypical hypernym,
and cactus is a prototypical hyponym.

In Chapter~\ref{ch:hpm}, we dive deeper into a hypernymy detection model which
is known to exhibit strong prototypicality behavior. We propose a novel
analysis of this model by interpreting the model within the framework of
{\em context vectors}, and show it learns to identify distributional contexts
related to Hearst patterns. We propose a novel model based on this behavior,
which extends a prototypicality classifier using an iterative procedure
similar to Principal Component Analysis (PCA). We evaluate our model in
hypernymy detection, general lexical entailment detection, and lexical
relationship prediction, and show our model matches or outperforms several
existing models in the literature.

In Chapter~\ref{ch:lexsub}, we propose two novel, unsupervised models for
the Lexical Substitution task, where one must propose a synonym of a target word in a given
    sentential context, which preserves the meaning of the sentence. We show our models quantitatively outperform
a similar model across three datasets of Lexical Substitution, especially
when models must propose Lexical Substitutes from the entire vocabulary.
An analysis of our models show they prefer to predict high-frequency words
as substitutes, enabling them to prefer correct substitutes over
misspelled alternatives.

In Chapter~\ref{ch:rte}, we integrate the findings of the previous chapters
into a real, end-to-end RTE system, showing that our contributions in lexical
semantics are useful in a complex task requiring sentence-level reasoning.
An error analysis of RTE shows the different components succeed and fail
in ways consistent with the findings of other chapters.

We conclude our thesis in Chapter~\ref{ch:conclusion}, summarizing our
contributions and findings, and proposing some directions for further
investigation.


\section{List of Thesis Contributions}

\noindent In this thesis, we make the following contributions:

\paragraph{Experimental Setup for Predicting Lexical Relationships}
We observe the importance of experimental setup in lexical relationship
    classification and hypernymy detection, noting that traditional train/test
    splits can lead to {\em lexical memorization} and overestimation of the
    lexical generalization.
We propose an alternative experimental setup for lexical relationship
    classification, where there is zero lexical overlap between the training
    and test sets. We find this experimental setup is considerably more
    difficult, and better measures the ability of models to generalize to
    novel lexical items.

\paragraph{The Asym model for Hypernymy Prediction}
We propose Asym, a supervised distributional model for hypernymy
    detection. We connect Asym to the Distributional Inclusion Hypothesis, which
    has motivated a large number of unsupervised models of hypernymy detection.
    We show the Asym model performs strongly in zero lexical overlap settings,
We analyze Asym within the context of the prototypicality framework
    proposed in prior work, and find it does not simply learn prototypicality
    behavior.

\paragraph{H-features for Lexical Relationship Prediction}
We analyze a lexical relationship classifier which is known to exhibit
    strong prototypicality behavior, wherein a model predicts \lit{sofa
    $\rightarrow$ animal}, simply because \lit{animal} is a prototypical
    hypernym. We consider its performance with respect to different
    distributional spaces, and use a novel technique to interpret the model
    with respect to distributional contexts. We observe that it predicts
    relationships using distributional contexts based on well-known Hearst
    patterns. We argue that this prototypicality behavior is interesting and
    valuable for the task of relationship prediction.

We propose a novel lexical relationship classifier called H-features,
    which exploits prototypicality behavior, and extends modeling power
    using an iterative PCA-like procedure for feature extraction. This new
    H-feature model obtains strong or state-of-the-art performance across
    nine datasets constituting a variety of linguistic relationships.
    We show that the model remains interpretable across this
    array of linguistic relationships, and learns to look for interesting
    distributional contexts beyond Hearst patterns.

\paragraph{The PIC and nPIC models for Lexical Substituion}
We propose {\em nPIC}, a novel unsupervised model for Lexical
    Substitution. Our model
    builds off of another simple model of Lexical Substitution, by replacing
    simple cosine similarity with an unnormalized dot product. Our Lexical
    Substitution model outperforms a comparable Lexical Substitution mode on
    three datasets. Our improvements are most pronounced in a difficult
    evaluation task, where a model must propose substitutes from the entire
    vocabulary.
    We also propose {\em PIC}, which includes additional parameters
    learned in a self-supervised manner. We find the additional parameters
    provide significant gains in objective performance on Lexical Substitution
    across all three datasets.

    We analyze our models of Lexical Substituion, and show our model
    prefers substitutions  with higher unigram frequencies, enabling it to
    disregard mispelled substitutes like \lit{colorfull (sic)} in favor of
    their correct counterparts. We also find that the additional parameters
    of the PIC model further exaggerate this unigram bias.

\paragraph{Integrations into an RTE system}
We evaluate lexical entailment classifiers within the framework of
    a complete end-to-end RTE system, and observe that strong performance
    in lexical entailment is a critical component for a successful RTE system.

We observe that the probability of a word pair entailing in RTE is
    non-monotonic in the cosine similarity between the words: the pair is
    more likely entailing the higher its cosine similarity, until the
    highest levels of cosine similarity, at which point words are more likely
    to be co-hyponyms and therefore non-entailing. We integrate this observation
    into the RTE system and show improvements at both the lexical level and
    RTE level.

Finally, we integrate the contributions from our chapters on
    Asym, H-features, and Lexical Substitution into the end-to-end RTE system.
    We find that many of the improvements observed in earlier chapters contribute
    to a higher score in the end-to-end RTE system, providing additional
    validation for the contributions of this thesis.


